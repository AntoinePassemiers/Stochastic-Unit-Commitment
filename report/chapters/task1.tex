\chapterwithsubtitle{Tâche 1}{Résoudre la relaxation linéaire de la Formulation SUC}
\vspace*{1.2cm}

\section{Relaxation linéaire}

La relaxation linéaire du modèle utilisé \footnote{A. Papavasiliou. Coupling Renewable Energy Supply with Deferrable Demand. PhD thesis,
University of California, Berkeley, 2011, pg 22} peut être formulée ainsi:\\\\

\begin{tabularx}{\textwidth}{l X r r}
(RL) \hspace{1cm} min \ $\sum\limits_{g \in G} \sum\limits_{s \in S} \sum\limits_{t \in T} \pi_s (K_g u_{gst} + S_g v_{gst} + C_g p_{gst})$ & & (3.20) \\\\
t.q. & \\\\
$\sum\limits_{l \in LI_n} e_{lst} + \sum\limits_{g \in G_n} p_{gst} = D_{nst} + \sum\limits_{l \in LO_n} e_{lst}$ & $\forall n \in N, s \in S, t \in T$ & (3.21) \\
$e_{lst} = B_{ls} (\theta_{nst} - \theta_{mst})$ & $\forall l = (m, n) \in L, s \in S, t \in T$ & (3.22) \\
$e_{lst} \le TC_l$ & $\forall l \in L, s \in S, t \in T$ & (3.23) \\
$-TC_l \le e_{lst}$ & $\forall l \in L, s \in S, t \in T$ & (3.24) \\
$p_{gst} \le P_{gs}^{+} u_{gst}$ & $\forall g \in G, s \in S, t \in T$ & (3.25) \\
$P_{sg}^{-} u_{gst} \le p_{gst}$ & $\forall g \in G, s \in S, t \in T$ & (3.26) \\
$p_{gst} - p_{gs,t-1} \le R_g^{+}$ & $\forall g \in G, s \in S, t \in T$ & (3.27) \\
$p_{gs,t-1} - p_{gst} \le R_g^{-}$ & $\forall g \in G, s \in S, t \in T$ & (3.28) \\
$\sum\limits_{q=t-UT_g+1}^{t} z_{gq} \le w_{gt}$ & $\forall g \in G_s, t \ge UT_g$ & (3.29) \\
$\sum\limits_{q=t+1}^{t+DT_g} z_{gq} \le 1 - w_{gt}$ & $\forall g \in G_s, t \le N - DT_g$ & (3.30) \\
$\sum\limits_{q=t-UT_g+1}^{t} v_{gsq} \le u_{gst}$ & $\forall g \in G_f, t \ge UT_g$ & (3.31) \\
$\sum\limits_{q=t+1}^{t+DT_g} v_{gsq} \le 1 - u_{gst}$ & $\forall g \in G_f, t \le N - DT_g$ & (3.32) \\
$z_{gt} \le 1$ & $\forall g \in G_s, t \in T$ & (3.33) \\
$v_{gst} \le 1$ & $\forall s \in S, t \in T$ & (3.34) \\
$z_{gt} \ge w_{gt} - w_{g,t-1}$ & $\forall g \in G_s, t \in T$ & (3.35) \\
$v_{gst} \ge u_{gst} - u_{gs,t-1}$ & $\forall g \in G_f, s \in S, t \in T$ & (3.36) \\
$\pi_s u_{gst} = \pi_s w_{gt}$ & $\forall g \in G_s, s \in S, t \in T$ & (3.37) \\
$\pi_s v_{gst} = \pi_s z_{gt}$ & $\forall g \in G_s, s \in S, t \in T$ & (3.38) \\
$p_{gst} \ge 0, \ 0 \le u_{gst} \le 1$ & $\forall g \in G, s \in S, t \in T$ & (3.39) \\
$z_{gt} \ge 0, \ 0 \le w_{gt} \le 1$ & $\forall g \in G_s, t \in T$ & (3.40) \\
\end{tabularx}

\vspace{2cm}

\subsection{Incertitudes du modèle}

\begin{itemize}
    \item Les contraintes (3.21) d'équilibre de marché imposent que la production d'énergie satisfasse la demande étant donné les niveaux de puissance
    assignés aux lignes de transmission.
    \item Les contraintes (3.22) sont obtenues à partir de la première loi de Kirchhoff (loi des noeuds) et de la seconde loi de Kirchhoff
    (loi des mailles). La susceptance y sert à modéliser la puissance lorsque les lignes sont parcourues par un courant continue.
    Les lignes du réseau peuvent être soumises à des contingences, symbolisées par les susceptances des lignes.
    Lorsqu'une ligne de transmission $l$ est mise hors service dans un scénario $s$, la susceptance associée $B_{ls}$ est fixée à 0.
    \item Les contraintes de types (3.25) et (3.26) tiennent compte des contingences liées aux générateurs eux-mêmes.
    Dans tout scénario $s$ où un générateur $g$ est hors usage, les constantes $P_{gs}^{+}$ et $P_{gs}^{-}$ sont fixées à 0 afin de forcer
    la production à 0.
\end{itemize}

\subsection{Non-anticipativité}

Les contraintes (3.37) et (3.38) de non-anticipativité permettent de forcer la planification des générateurs lents établie lors de la seconde phase
à respecter celle effectuée pour ces mêmes générateurs lors de la première phase. En effet les contingences peuvent être énumérées (d'où la présence
de scénarios) mais ne peuvent avoir de caratère certain. L'engagement des générateurs lents doit donc être le même dans tous les scénarios,
d'où les contraintes d'égalité. Étant donné que les variables $w_{gt}$ et $z{gt}$ sont définies pour les générateurs lents
et les variables $u_{gst}$ et $v_{gst}$ pour tous les générateurs, $u_{gst}$ et $v_{gst}$ sont redondantes pour les générateurs lents.

Cette redondance n'est justifiée que par l'utilisation de la relaxation lagrangienne et la dualisation des contraintes de non-anticipativité, car
elle permet de borner les sous-problèmes produits par la décomposition lagrangienne. En l'absence d'une telle relaxation il convient alors de retirer 
de la modélisation les variables $u_{gst}$ et $v_{gst}$ associées aux générateurs lents. Puisque nous avons gardé la même approche qu'Anthony Papavasiliou
pour la décomposition lagrangienne, toutes les variables ont été gardées.

\subsection{Démarrage des générateurs et contraintes temporelles}

Les contraintes de type (3.35) et (3.36) couplent des variables liées à des périodes de temps adjacentes.
Par exemple, les contraintes de type (3.35) sont définies $\forall t \in T$. Or les valeurs des variables
$w_{gt}$ ne sont pas connues pour $t = 0$. Nous n'avons pas fait l'hypothèse que les générateurs en question
sont éteints en $t = 0$, et simplement retiré la première contrainte. Les contraintes sont donc définies
$\forall t \in T \backslash \{0\}$, ce qui simplifie le problème et allège légèrement les temps d'exécution.
