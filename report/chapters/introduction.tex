\chapter{Exemples d'appel au programme}
\vspace*{1.2cm}

\begin{lstlisting}[language=bash]
  $ # Résolution du primal
  $ python main.py <path_to_instance>
  $ # Résolution de la relaxation linéaire
  $ python main.py <path_to_instance> --relax
  $ # Résolution de la relaxation linéaire + arrondi
  $ python main.py <path_to_instance> --relax --round
  $ # Décomposition lagrangienne et méthode du sous-gradient
  $ python main.py <path_to_instance> --decompose
  $ python main.py <path_to_instance> --decompose --nar 6 --epsilon 0.01 --alpha0 2000 --rho 0.96
\end{lstlisting}

Les différents paramètres de l'algorithme du sous-gradient sont les suivants:
\begin{itemize}
    \item $n_{ar}$ (nar): le nombre d'itérations du sous-gradient à effectuer avant de commencer à appliquer
    l'heuristique et obtenir des solutions primales faisables.
    \item $\epsilon$ (epsilon): Seuil de convergence / saut de dualité en dessous duquel l'algorithme est considéré comme
    ayant convergé. Lorsque $(UB - LB) / UB  < \epsilon$, l'algorithme s'arrête.
    \item $\alpha_0$ (alpha0): Le pas initial du sous-gradient.
    \item $\rho$ (rho): Le facteur de diminution du pas (si aucune solution primale admissible n'a encore été trouvée).
\end{itemize}