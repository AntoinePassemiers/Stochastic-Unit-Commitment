\chapterwithsubtitle{Tâche 4}{Développer une heuristique basée sur la solution obtenue par la relaxation de la tâche précédente}
\vspace*{1.2cm}

Contrairement à ce qui a été suggéré dans l'énoncé et malgré l'utilité que peuvent avoir les multiplicateurs lagrangiens
dans la recherche d'une solution primale admissible, nous avons choisi d'exploiter exclusivement l'information
contenue dans l'historique des solutions primales non-admissibles obtenues avec la méthode du sous-gradient.
En effet, beaucoup d'auteurs dont \citep{doi:10.1137/S1052623498332336} et \citep{Zhuang1988}
suggèrent de rapprocher d'avantage la solution courante de la zone admissible du primal en calculant une combinaison
convexe des différentes solutions obtenues pour le primal.


\subsubsection{Calcul du pas sur base d'une borne supérieure}

Une façon naïve de calculer une borne supérieure rapidement est de fixer les variables
$u_{gst}, v_{gst}, p_{gst} \ \forall g \in G, s \in S, t \in T$ à leurs valeurs maximales respectives.
Il est assez clair que la solution correspondante devient infaisable et que seules les contraintes 
$(3.25), (3.33), (3.34), (3.39), (3.40)$ sont garanties d'être satisfaites. Par exemple, certaines contraintes $(3.36)$ cessent d'être satisfaites
car il n'est pas possible d'allumer un générateur deux fois d'affilée. Cependant, la valeur de l'objectif constitue une borne supérieure
car toute solution faisable nécessite d'abaisser la valeur d'une des variables $u_{gst}, v_{gst}, p_{gst} \ \forall g \in G, s \in S, t \in T$
afin de satisfaire les contraintes restantes. En revanche cette borne est très mauvaise et \textbf{peut valoir plusieurs fois la valeur de la solution optimale}.
Cette borne $WUB$ est calculée ainsi:
\begin{equation}
    WUB = \sum\limits_{g \in G} \sum\limits_{s \in S} \sum\limits_{t \in T} \pi_s (K_g + S_g + C_g P_{gs}^{+})
\end{equation}

TODO
