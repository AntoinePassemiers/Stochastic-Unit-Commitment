\chapterwithsubtitle{Tâche 4}{Développer une heuristique basée sur la solution obtenue par la relaxation de la tâche précédente}
\vspace*{1.2cm}

Contrairement à ce qui a été suggéré dans l'énoncé et malgré l'utilité que peuvent avoir les multiplicateurs lagrangiens
dans la recherche d'une solution primale admissible, nous avons choisi d'exploiter exclusivement l'information
contenue dans l'historique des solutions primales non-admissibles obtenues avec la méthode du sous-gradient.
En effet, beaucoup d'auteurs dont \citep{doi:10.1137/S1052623498332336} et \citep{Zhuang1988}
suggèrent de rapprocher d'avantage la solution courante de la zone admissible du primal en calculant une combinaison
convexe des différentes solutions obtenues pour le primal.

\section{Séquences ergodiques}

Nous définissons une séquence ergodique comme étant une série $\{\overline{x}^k\}$ où l'élement $\overline{x}^k$
est calculé ainsi \citep{Aldenvik}:
\begin{equation}
    \overline{x}^k = \sum\limits_{s=0}^{t-1} \mu_s^t x^s \ , \ \sum\limits_{s=0}^{t-1} \mu_s^t = 1 \ , \ \mu_s^t \le 0 \ , \ s = 0, \ldots, t-1
\end{equation}
où $x^s$ est une solution primale infaisable obtenue à l'itération $s$ de l'algorithme du sous-gradient et $\mu_s^t x^s$ est un coefficient
de la combinaison convexe $\overline{x}^k$. La nouvelle solution primale obtenue à l'itération $t$ est donc bien une combinaison convexe 
des solutions déjà obtenues.

Andrea Simonetto et Hadi Jamali-Rad~\citep{Simonetto2016} fournissent une preuve de convergence forte de cette série.
En revanche, la combinaison des variables binaires de notre problème ont de grandes chances de \textbf{générer des valeurs fractionnaires} pour
celles-ci. Il a été montré que ce nombre de valeurs fractionnaires est fortement réduit lorsque le nombre d'unités dans le problème
est significativement supérieur au nombre de périodes de temps. Cependant ceci n'est pas notre cas et il est nécessaire de recourir à une 
méthode de recherche locale afin de trouver une solution respectant les contraintes d'intégralité.
La convergence forte n'est pas assurée mais cependant dans le cas d'un problème entier (binaire) mixte (ce qui est le cas), la série converge vers 
un point de l'enveloppe convexe de la zone admissible.
Comme méthode de recherche locale, nous avons simplement utilisé \textbf{l'heuristique implémentée dans le cadre de la tâche 2} car nous l'avons
justement spécialement conçue pour les solutions contenant des valeurs fractionnaires.

\section{Bornes supérieures}

L'algorithme tel qu'implémenté dans la tâche 3 a été modifié afin de calculer une borne supérieure sur base de solution primales admissibles.
Cependant, au début de la méthode du sous-gradient, les solutions duales trouvées génèrent des solutions primales
correspondantes fort éloignée de la zone admissible: il y a donc de plus grandes choses de voir l'heuristique échouer.
Nous introduisons un paramètre \textit{$n_{ar}$} correspondant au nombre d'itérations du sous-gradient avant de commencer à utiliser
l'heuristique pour reconstruire une solution du primal. Lorsqu'une première solution admissible est trouvée, la borne supérieure est abaissée
à la valeur de l'objectif trouvé et le pas du sous-gradient est calculé sur base des bornes supérieures et inférieures.
Tant qu'aucune solution admissible n'est disponible, le pas est calculé via la suite géométrique décrite à la fin de la tâche 3.

\subsubsection{Calcul du pas sur base d'une borne supérieure}

Une façon naïve de calculer une borne supérieure rapidement est de fixer les variables
$u_{gst}, v_{gst}, p_{gst} \ \forall g \in G, s \in S, t \in T$ à leurs valeurs maximales respectives.
Il est assez clair que la solution correspondante devient infaisable et que seules les contraintes 
$(3.25), (3.33), (3.34), (3.39), (3.40)$ sont garanties d'être satisfaites. Par exemple, certaines contraintes $(3.36)$ cessent d'être satisfaites
car il n'est pas possible d'allumer un générateur deux fois d'affilée. Cependant, la valeur de l'objectif constitue une borne supérieure
car toute solution faisable nécessite d'abaisser la valeur d'une des variables $u_{gst}, v_{gst}, p_{gst} \ \forall g \in G, s \in S, t \in T$
afin de satisfaire les contraintes restantes. En revanche cette borne est très mauvaise et \textbf{peut valoir un multiple de l'objectif de la solution optimale}.
Cette borne $WUB$ est calculée ainsi:
\begin{equation}
    WUB = \sum\limits_{g \in G} \sum\limits_{s \in S} \sum\limits_{t \in T} \pi_s (K_g + S_g + C_g P_{gs}^{+})
\end{equation}

Cette borne supérieure a été utilisée lors de l'implémentation du sous-gradient de la tâche 3 et a été abandonnée au profit de la suite géométrique et
des objectifs des solutions primales trouvées heuristiquement.

\section{Vue générale de la méthode}

\begin{algorithm}[H]
\caption{Méthode du sous-gradient pour le problème SUC}
\begin{algorithmic}[1]
\Procedure{solve\_with\_subgradient}{}
\State $k \leftarrow 0$
\State $UB = \leftarrow WUB, \ LB = -\infty$ 
\State $\mu_{gst} = 0 \ , \ nu_{gst} = 0 \ \ \forall g \in G, s \in S, t \in T$
\While {$\frac{(UB - LB)}{UB} \le \epsilon$}
\State Résoudre les sous-problèmes $P2$ et $P1_s \ \forall s \in S$
\State $L_k \leftarrow $ \ somme des objectifs des différents sous-problèmes
\If {$L_k = LB$}
\State $\lambda \leftarrow \lambda / 2$
\EndIf
\If {$L_k > LB$}
\State $LB = L_k$
\EndIf
\If {$k > n_{ar}$}
\State Somme ergodique $\overline{x}^k$ de toutes les solutions primales infaisables obtenues
\State Obtenir une solution primale admissible de manière heuristique depuis $\overline{x}^k$
\EndIf
\If {$\hat{L} < UB$}
\State $UB = \hat{L}$
\EndIf
\If {$k > n_{ar}$ et une solution admissible a été trouvée}
\State $\alpha_k = \frac{\lambda (\hat{L} - L^k)}{\sum\limits_{g \in G_s} \sum\limits_{s \in S} \sum\limits_{t \in T} \big(\pi_s^2 \ (u_{gst}^k - w_{gt}^k)^2 + \pi_s^2 \ (v_{gst}^k - z_{gt}^k)^2\big)}$
\Else
\State $\alpha_k = \alpha_0 \ \rho^k$
\EndIf
\State Mise à jour des multiplicateurs $\mu_{gst}, \nu{gst} \ \ \forall g \in G, s \in S, t \in T$
\State $k \leftarrow k + 1$
\EndWhile
\EndProcedure
\end{algorithmic}
\end{algorithm}