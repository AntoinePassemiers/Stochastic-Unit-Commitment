\chapterwithsubtitle{Tâche 3}{Résoudre une relaxation Lagrangienne de la Formulation SUC}
\vspace*{1.2cm}

\section{Décomposition lagrangienne}

Le choix des contraintes relâchées est le même que celui effectué par A. Papavasiliou:
seules les contraintes de non-anticipitavité ont été dualisées. Les deux raisons
principales sont que d'une part elles compliquent fortement le problème à résoudre
et que d'autre part, elles sont les seules contraintes à coupler les différents
scénarios entre eux. Sans elles, \textit{il est désormais possible de décomposer le problème
suivant ses différents scénarios}.

Enfin, la relaxation de ses contraintes seules est suffisante pour diminuer significativement
le temps d'exécution du solveur. En particulier, la somme des temps d'exécution sur les
différents sous-problèmes créés est fortement inférieure au temps d'exécution sur le problème
primal, et ceci est d'autant plus valable pour les instances de grande taille.

Le dual lagrangien est donc obtenu en reprenant la fonction objectif du primal et en y
dualisant les contraintes de non-anticipativité $(3.37)$ et $(3.38)$:

\begin{align*}
    \mathcal{L} & = \sum\limits_{g \in G} \sum\limits_{s \in S} \sum\limits_{t \in T} \pi_s (K_s u_{gst} + S_g v_{gst} + C_g p_{gst}) + \sum\limits_{g \in G_s} \sum\limits_{s \in S} \sum\limits_{t \in T} \pi_s (\mu_{gst} (u_{gst} - w_{gt}) + \nu_{gst} (v_{gst} - z_{gt})) \\
    & = \sum\limits_{s \in S} \Big(\sum\limits_{g \in G} \sum\limits_{t \in T} (K_s u_{gst} + S_g v_{gst} + C_g p_{gst}) + \sum\limits_{g \in G_s} \sum\limits_{t \in T} \pi_s (\mu_{gst} u_{gst} + \nu_{gst} v_{gst})\Big) \\
    & - \sum\limits_{g \in G_s} \sum\limits_{s \in S} \sum\limits_{t \in T} \pi_s (\mu_{gst} w_{gt} + \nu_{gst} z_{gt}) \\
\end{align*}

Nous observons qu'en réarrangeant les termes de la fonction il est possible de l'exprimer sous la forme d'une somme de plusieurs objectifs, 
dont un est indépendant du scénario et chacun des autres est assigné à un scénario $s$.
La décomposition lagrangienne se fait alors comme suit:

\begin{itemize}
    \item \textbf{Les sous-problèmes P1$_{\text{s}}$} concernent la plannification des générateurs rapides.
    Le sous-problème $P1_s$ requiert de plannifier les générateurs rapides dans le cadre du scénario $s$.
    La fonction objectif est donnée par \citep{Papavasiliou2013}:
    \begin{equation}
        \min \ \sum\limits_{g \in G} \sum\limits_{t \in T} (K_s u_{gst} + S_g v_{gst} + C_g p_{gst}) + \sum\limits_{g \in G_s} \sum\limits_{t \in T} \pi_s (\mu_{gst} u_{gst} + \nu_{gst} v_{gst})
    \end{equation}
    \item \textbf{Le sous-problème P2} concerne la plannification des générateurs lents et ne prend donc pas
    en considération les différents scénarios. La valeur de son objectif est quant à elle donnée par:
    \begin{equation}
        \min \ - \sum\limits_{g \in G_s} \sum\limits_{s \in S} \sum\limits_{t \in T} \pi_s (\mu_{gst} w_{gt} + \nu_{gst} z_{gt})
    \end{equation}
\end{itemize}

La valeur de l'objectif du dual lagrangien est donc exprimé sous la forme d'une somme des objectifs des différents sous-problèmes
issus de cette décomposition. La tâche consiste à présent à optimiser le dual lagrangien afin de rapprocher la solution duale de la
zone admissible du primal.

\section{Optimisation du dual lagrangien}

\subsection{Choix de l'algorithme}

Comme expliqué par A. Papavasiliou dans une présentation de 2016 \citep{Asynchronous},
il est possible que certains sous-problèmes $P1_s$ prennent considérablement plus de temps que le autres pour être résolus, \textit{jusqu'à
75 fois le temps pris par le sous-problème le plus rapide}. Ceci crée des goulots d'étranglement lors de l'optimisation du dual,
qui peuvent cependant être résolus par l'utilisation d'un algorithme d'optimisation asynchrone.
En particulier, l'\textit{algorithme de descente par blocs} a été présenté et permet de mettre à jour un sous-ensemble des multiplicateurs
lagrangiens en calculant une partie du sous-gradient à la fois. Ce fragment est le sous-ensemble de composantes associées à un seul
scénario. L'algorithme se distingue donc des algorithmes de descente par coordonnée car plusieurs variables duales sont mis à jour
à la fois.

Nous avons eu la chance de ne pas observer de tels ralentissements pour les instances fournies et les temps de résolution des différents
sous-problèmes étaient fort similaires. En conséquence, nous avons fait le choix d'utiliser l'\textit{algorithme du sous-gradient} car il est
plus simple à implémenter que l'algorithme de descente par blocs.
Le sous-gradient $d^k$ associés aux multiplicateurs lagrangiens à l'itération $k$ est donné par:
\begin{align}
    d_{\mu_{gst}}^k = & \ \pi_s (w_{gt}^k - u_{gst}^k) \ \  \forall g \in G_s, s \in S, t \in T \\
    d_{\nu_{gst}}^k = & \ \pi_s (z_{gt}^k - v_{gst}^k) \ \  \forall g \in G_s, s \in S, t \in T
\end{align}
où $d_{\mu}^k$ est le partie du sous-gradient associée aux contraintes $(3.37)$ et $d_{\nu}^k$ est celle associée aux contraintes $(3.38)$.
Il est important de noter que \textbf{les multiplicateurs lagrangiens sont mis à jour à l'aide de l'équation décrite dans cette même présentation
et non celle présentée dans la thèse} \citep{Papavasiliou12couplingrenewable} (il s'agit selon nous d'une simple erreur de signe).
Les multiplicateurs lagrangiens sont donc calculés d'itération en itération de la façon suivante:
\begin{align}
    \mu_{gst}^{0} & = 0 \ \  \forall g \in G_s, s \in S, t \in T \\
    \nu_{gst}^{0} & = 0 \ \  \forall g \in G_s, s \in S, t \in T \\
    \mu_{gst}^{k+1} & = \ \mu_{gst}^{k} - \alpha_k d_{\mu_{gst}}^k \ \  \forall g \in G_s, s \in S, t \in T \\
    \nu_{gst}^{k+1} & = \ \nu_{gst}^{k} - \alpha_k d_{\nu_{gst}}^k \ \  \forall g \in G_s, s \in S, t \in T
\end{align}
L'étape suivante est donc de trouver un pas $\alpha_k$ assurant une convergence de l'algorithme.


\subsection{Calcul du pas}

Un pas de déplacement fortement utilisé en pratique, décrit par Fisher et Held notamment, repris par Papavasiliou, est décrit ainsi:
\begin{equation}
    \alpha^k = \frac{\lambda (\hat{L} - L^k)}{\sum\limits_{g \in G_s} \sum\limits_{s \in S} \sum\limits_{t \in T} \big(\pi_s^2 \ (u_{gst}^k - w_{gt}^k)^2 + \pi_s^2 \ (v_{gst}^k - z_{gt}^k)^2\big)}
\end{equation}
où $\lambda$ est un paramètre constant, $\hat{L}$ est une borne supérieure sur la solution optimale du primal et $L^k$ est la somme des objectifs des sous-problèmes
à l'itération $k$. Il reste donc le problème de trouver une borne supérieure raisonnable sur la valeur de la solution optimale,
et que l'on puisse calculer rapidement. Il reste encore à calculer une borne supérieure, qui peut être trouvée notamment avec une solution primale faisable.
Le problème est que trouver une solution faisable constitue déjà une difficulté. Puisque l'obtention d'une solution primale faisable
concerne la tâche 4 du projet, nous avons décidé de \textbf{nous affranchir de ce problème dans le cadre de la tâche 3}.
Nous n'avons donc pas pris en considération l'idée de Papavasiliou consistant en l'obtention d'une solution faisable par résolution
du problème de répartition économique (\textit{economic dispatch}).
Deux autres raisons à ce choix est qu'une telle solution ne peut être obtenue que vers les dernières itérations du sous-gradient
(nous laissant donc sans borne supérieure durant la majorité du temps d'exécution), et que la résolution des différents sous-problèmes
de répartition économique (un par scénario) ralentit le sous-gradient, malgré le fait que les différents sous-problèmes peuvent être résolus en parallèle.
Pour cette tâche nous avons opté pour une alternative plus simple inspirée de \citep{Zhuang1988}.
La méthode a été montrée comme efficace sur le problème de \textit{unit commitment} et calcule le pas de déplacement uniquement
sur base de deux paramètres de contrôle et du numéro de l'itération courante. Dans l'article en question le pas est mis à jour de la façon suivante:
\begin{equation}
    \alpha^k = \frac{1}{\alpha + \beta k} \ , \  \alpha, \beta > 0
\end{equation}
Dans l'implémentation de notre sous-gradient, nous avons plutôt utilisé la formule suivante:
\begin{equation}
    \alpha_k = \alpha_0 \rho^k
\end{equation}
où $\alpha_0$ et $\rho$ doivent être choisis suffisamment grands afin d'assurer la convergence de l'algorithme.